\Lecture{February 11, 2025}

\section{Adjoint Functors}

\begin{definition*}
	Functors \( F: \C \to \ct{D} \) and \( G: \ct{D} \to \C \) are called \emdef{adjoint} (\( F \) is \emdef{left adjoint} of \( G \) and \( G \) is \emdef{right adjoint} of \( F \); we write \( F \adj G \)) if there is a binatural bijection
	\[
		\ct{D}(F(X), Y) \cong \C(X, G(Y)).
	\]
\end{definition*}

\Heading{Motivation}

Let \( k \) be a field and \( S \) be a set. We write
\[
	k^{(S)} = k^{\oplus S} := \bigoplus_{s \in S} k.
\]
If \( U: \ctrm{Vect} \to \ctrm{Sets} \) is the forgetful functor, then
\[
	\Hom_{\ctrm{Vect}}(k^{(S)}, V) \cong \Hom_{\ctrm{Sets}}(S, U(V)).
\]
It turns out that the isomorphism is also binatural, implying that \( k^{(-)} \) is left adjoint to \( U \).

\Heading{Binaturality}

\begin{enumerate}
	\item Naturality in the first argument
		\[
			\begin{tikzcd}{X'} && {\ct{D}(F(X), Y)} &&& {\C(X, G(Y))} \\
				\\
				X && {\ct{D}(F(X'), Y)} &&& {\C(X', G(Y))}
				\arrow["h"', from=1-1, to=3-1]
				\arrow["{\varphi^{-1}_{X, Y}}", from=1-3, to=1-6]
				\arrow["{\ct{D}(F(h), Y)}"', from=1-3, to=3-3]
				\arrow["{\C(h, G(Y))}", from=1-6, to=3-6]
				\arrow["{\varphi^{-1}_{X', Y}}"', from=3-3, to=3-6]
			\end{tikzcd}
		\]
	\item Naturality in the second argument
		\[
			\begin{tikzcd}
				Y && {\ct{D}(F(X), Y)} &&& {\C(X, G(Y))} \\
				\\
				{Y'} && {\ct{D}(F(X), Y')} &&& {\C(X, G(Y'))}
				\arrow["h"', from=1-1, to=3-1]
				\arrow["{\varphi_{X, Y}}", from=1-3, to=1-6]
				\arrow["{\ct{D}(F(X), h)}"', from=1-3, to=3-3]
				\arrow["{\C(X, G(h))}", from=1-6, to=3-6]
				\arrow["{\varphi_{X, Y'}}"', from=3-3, to=3-6]
			\end{tikzcd}
		\]
\end{enumerate}

Given the bijection
\[
	\varphi_{X,Y}: \ct{D}(F(X), Y) \cong \C(X, G(Y)),
\]
it is convenient to speak of transposes: for a morphism \( g: F(X) \to Y \), its \emdef{transpose} is
\[
	\bar{g} := \varphi_{X,Y}(g): X \to G(Y);
\]
conversely, for \( f: X \to G(Y) \), its \emdef{transpose} is
\[
	\bar{f} := \varphi^{-1}_{X,Y}(f): F(X) \to Y.
\]
These assignments are mutually inverse, so \( \bar{\bar{g}} = g \) and \( \bar{\bar{f}} = f \).

\newpage

Then the two naturalities can be phrased succinctly in this language:
\begin{enumerate}
	\item For every \( h: X' \to X \), transposition commutes with precomposition:
		\[
			\overline{f h} \;=\; \bar{f}\, F(h).
		\]
	\item For every \( h: Y \to Y' \), transposition commutes with postcomposition:
		\[
			\overline{h g} \;=\; G(h)\, \bar{g}.
		\]
\end{enumerate}

Let now \( X \in \Ob(\C) \) and set \( Y = F(X) \). Define
\[
	\eta_X := \bar{1}_{F(X)} : X \to GF(X).
\]
This map is called the \emdef{unit} of the adjunction: it is the image, under the adjunction bijection, of the identity on \( F(X) \), and it is natural in \( X \) by binaturality of \( \varphi \). In particular, for any \( h : X' \to X \) one has the naturality relation
\[
	GF(h)\, \eta_{X'} \;=\; \eta_X\, h.
\]

Dually, define
\[
	\varepsilon_Y := \bar{1}_{G(Y)} : FG(Y) \to Y.
\]
This map is called the \emdef{counit} of the adjunction: it is the image, under \( \varphi^{-1} \), of the identity on \( G(Y) \), and it is natural in \( Y \) by binaturality of \( \varphi \). The unit and counit satisfy the \emdef{triangle identities}, which express that transposition and transposition back act as the identity on both sides of the adjunction:
\[
	\varepsilon_{F(X)} \circ F(\eta_X) \;=\; 1_{F(X)},
	\qquad
	G(\varepsilon_Y) \circ \eta_{G(Y)} \;=\; 1_{G(Y)}.
\]

Using \( \eta \) and \( \varepsilon \), the transposes admit explicit formulas. For \( g : F(X) \to Y \), one has
\[
	\bar{g} : X \tox{\eta_X} GF(X) \tox{G(g)} G(Y).
\]
Conversely, for \( f : X \to G(Y) \), its transpose is the composite
\[
	\bar{f} : F(X) \tox{F(f)} FG(Y) \tox{\varepsilon_Y} Y.
\]
By the \textbf{triangle identities}, these constructions are inverse to one another: \( \bar{\bar{g}} = g \) and \( \bar{\bar{f}} = f \).

\begin{lemma*}[Triangular identities]
	If \( F \adj G \), then the following diagrams are commutative:
	\[
		\begin{tikzcd}
			F &&& FGF && G &&& GFG \\
			\\
			&&& F &&&&& G
			\arrow["{F\eta}", from=1-1, to=1-4]
			\arrow["{1_F}"', from=1-1, to=3-4]
			\arrow["{\varepsilon F}", from=1-4, to=3-4]
			\arrow["{\eta_G}", from=1-6, to=1-9]
			\arrow["{1_G}"', from=1-6, to=3-9]
			\arrow["{G\varepsilon}", from=1-9, to=3-9]
		\end{tikzcd}
	\]
\end{lemma*}
\begin{proof}
	The following establishes the first identity and finishes the proof (by duality):
	\[
		1_{F(X)} = \overline{\eta_X} = \overline{1_{GF(X)} \eta_X} \overset{(2)}{=} \overline{1_{GF(X)}} F(\eta_X) = \varepsilon_{F(X)} F(\eta_X).
	\]
\end{proof}

\begin{theorem*}
	Let \( F: \C \to \ct{D} \), \( G: \ct{D} \to \C \). There is a one-to-one correspondence between
	\begin{itemize}
		\item[(a)] adjunctions between \( F \) and \( G \);
		\item[(b)] natural transformations \( \eta: 1_{\C} \to GF,~ \varepsilon: FG \to 1_{\ct{D}} \) satisfying triangular identities.
	\end{itemize}
\end{theorem*}
\begin{proof}
	It is left to define a function from (b) to (a). Given \( g: F(X) \to Y \) and \( f: X \to G(Y) \), we set
	\[
		\bar{g} := (X \tox{\eta_X} GF(X) \tox{G(g)} G(Y)),
		\qquad
		\bar{f} := (F(X) \tox{F(f)} FG(Y) \tox{\varepsilon_Y} Y).
	\]
	We claim that maps \( g \to \bar{g} \) and \( f \to \bar{f} \) are inverse to each other. Indeed,
	\[
		\begin{tikzcd}{F(X)} &&& {FGF(X)} &&& {FG(Y)} \\
			\\
			&&& {F(X)} &&& Y
			\arrow["{F(\eta_X)}", from=1-1, to=1-4]
			\arrow["1"', from=1-1, to=3-4]
			\arrow["{FG(g)}", from=1-4, to=1-7]
			\arrow["{\varepsilon_{F(X)}}", from=1-4, to=3-4]
			\arrow["{\varepsilon_Y}", from=1-7, to=3-7]
			\arrow["g"', from=3-4, to=3-7]
		\end{tikzcd}
	\]

	This follows from the following computation:
	\[
		g = \varepsilon_Y FG(g) F(\eta_X) = \varepsilon_Y F(G(g) \eta_X) = \eta_Y F(\bar{g}) = \bar{\bar{g}}.
	\]

	Finally, the chain of equalities
	\[
		\overline{hg} = G(hg) \eta_X = G(h) G(g) \eta_X = G(h) \bar{g}
	\]
	verifies the naturality conditions and finishes the proof.
\end{proof}

\begin{proposition*}
	If a functor \( F: \C \to \ct{D} \) admits a right adjoint, then this adjoint is unique up to isomorphism. Moreover, \( F \) admits a right adjoint if and only if the functor
	\[
		\ct{D}(F(-), Y): \C^\op \to \ctrm{Sets}
	\]
	is representable for every \( Y \in \Ob(\ct{D}) \).
\end{proposition*}

\begin{proof}
	Consider the functor
	\[
		F_*: \ct{D} \too [\C^\op, \ctrm{Sets}],
		\quad 
		F_*(Y) := \ct{D}(F(-), Y),
	\]
	and recall the Yoneda embedding \( h^{(-)}: \C \to [\C^\op, \ctrm{Sets}] \), where \( h^{X} = \C(-, X) \).

	\Heading{Uniqueness}

	If \( G, G': \ct{D} \to \C \) are both right adjoint to \( F \), then for each \( Y \) there are natural bijections
	\[
		\C(-, GY) \cong \ct{D}(F(-), Y) \cong \C(-, G'Y),
	\]
	natural in the \( \C \)-variable. By Yoneda, this yields natural isomorphisms \( GY \cong G'Y \), hence \( G \cong G' \).

	\Heading{Existence}

	The existence of a right adjoint to \( F \) is equivalent to a factorization of \( F_* \) through \( h^{(-)} \):
	\[
		\begin{tikzcd}
			\ct{D} \arrow[rr, "F_*"] \arrow[dr, dashed, "G_0"'] && {[\C^\op, \ctrm{Sets}]} \\
			& \C \arrow[ur, "h^{(-)}"'] &
		\end{tikzcd}
	\]
	that is, to the existence of isomorphisms
	\[
		F_*(Y) \cong h^{G_0(Y)} \quad \text{or equivalently} \quad
		\ct{D}(F(-), Y) \cong \C(-, G_0(Y)),
	\]
	natural in both variables. If \( F \adj G \), then for every \( Y \) the functor \( \ct{D}(F(-), Y) \) is represented by \( G(Y) \), so \( F_* \cong h^{(-)} \circ G \). Conversely, assume each \( F_*(Y) \) is representable. Choose for each \( Y \) an object \( G_0(Y) \in \Ob(\C) \) and a natural isomorphism
	\[
		\theta_Y: h^{G_0(Y)} \tox{ \cong} F_*(Y).
	\]
	For a morphism \( f: Y \to Y' \) in \( \ct{D} \), define \( G_0(f): G_0(Y) \to G_0(Y') \) to be the unique arrow corresponding under Yoneda to the natural transformation
	\[
		h^{G_0(Y)} \tox{\theta_Y} F_*(Y) \tox{F_*(f)} F_*(Y') \tox{\theta_{Y'}^{-1}} h^{G_0(Y')}.
	\]
	This makes \( G_0 \) a functor and the \( \theta_Y \) natural in \( Y \). Consequently, for all \( X \in \Ob(\C) \), \( Y \in \Ob(\ct{D}) \) there are natural bijections
	\[
		\C(X, G_0Y)
		\cong [\C^\op, \ctrm{Sets}]\bigl(h^{X}, h^{G_0Y}\bigr)
		\cong [\C^\op, \ctrm{Sets}]\bigl(h^{X}, F_*(Y)\bigr)
		\cong \ct{D}(FX, Y),
	\]
	natural in both \( X \) and \( Y \). Thus \( G_0 \) is a right adjoint to \( F \).
\end{proof}

\section{Terminal and Initial Objects}

\begin{definitions*}
	\item An object \( T \in \Ob(\C) \) is \emdef{terminal} if for all \( X \in \Ob(\C) \) there exists a unique map \( f: X \to T \).

	\item An object \( S \in \Ob(\C) \) is \emdef{initial} if for all \( X \in \Ob(\C) \) there exists a unique map \( f: S \to X \).

	\item An object \( Z \in \Ob(\C) \) is \emdef{zero} if \( Z \) is both initial and terminal.
	\item A map \( f: X \to Y \) is \emdef{zero} if \( f \) factors through the zero object.
\end{definitions*}

\begin{examples*}
	\item In \( \ctrm{Sets} \), \( S = \varnothing \) is initial, and \( T = \{\bullet\} \) is terminal.
	\item In \( \ctrm{R-Mod} \), \( Z = 0 \) is a zero object.
	\item In \( \ctrm{Rings} \), \( S = \bb{Z} \) is initial, and \( T = 0 \) is terminal.
\end{examples*}

\section{Comma Category}

\begin{definition*}
	Given a diagram of categories and functors
	\[
		\C \tox{F} \ca{E} \otx{G} \ct{D},
	\]
	the \emdef{comma category} \( (F \comma G) \) is defined by the following data:
	\begin{enumerate}[i)]
		\item Its objects are triples \( (C, h, D) \), where \( C \in \Ob(\C), D \in \Ob(\ct{D}) \), and \( h \in \ca{E}(F(C), G(D)) \).
		\item Its morphisms \( (C, h, D) \to (C', h', D') \) are pairs \( (f, g) \), where \( f \in \C(C, C'),~ g \in \ct{D}(D, D') \)
			are such the following diagram is commutative:
			\[
				\begin{tikzcd}{F(C)} &&& {G(D)} \\
					\\
					{F(C')} &&& {G(D')}
					\arrow["h", from=1-1, to=1-4]
					\arrow["{F(f)}"', from=1-1, to=3-1]
					\arrow["{G(g)}", from=1-4, to=3-4]
					\arrow["{h'}"', from=3-1, to=3-4]
				\end{tikzcd}
			\]
	\end{enumerate}
\end{definition*}

\begin{examples*}
	\item Given an object \( c \in \Ob(\C) \), the \emdef{slice category} is the comma category for the diagram
		\[
			\C \tox{1_{\C}} \C \otx{c} 1,
		\]
		where \( 1 \tox{c} \C \) is the constant functor with value \( c \). It is commonly denoted by
		\[
			(1_{\C} \comma c) \equiv (c \comma \C) \equiv \C \slice c.
		\]
	\item Analogously, the \emdef{coslice category} is defined by the diagram
		\[
			1 \tox{c} \C \otx{1_{\C}} \C,
		\]
		and is usually denoted by
		\[
			(c \comma 1_{\C}) \equiv (\C \comma c) \equiv c \slice \C.
		\]
	\item The diagram
		\[
			\C \tox{1_{\C}} \C \otx{1_{\C}} \C
		\]
		defines the \emdef{arrow category} denoted as follows:
		\[
			(1_{\C} \comma 1_{\C}) \equiv \ctrm{Arr}(\C) \equiv (\C \comma \C).
		\]
\end{examples*}