\Lecture{April 22, 2025}

\begin{definition*}
	Let \( f: A \to B \) be a chain map. The \emdef{cone} of \( f \) is the complex \( C(f) \) defined degreewise by
	\[
		C(f)_n := B_n \oplus A_{n-1} = B_n \oplus A[1]_n,
	\]
	with differential given, in the decomposition \( B_n \oplus A_{n-1} \to B_{n-1} \oplus A_{n-2} \), by the block matrix
	\[
		d^{C(f)}_n =
		\begin{pmatrix}
			d^B_n & -\,f_{n-1}   \\
			0     & -\,d^A_{n-1}
		\end{pmatrix}
		,
		\quad\text{i.e.}\quad
		d^{C(f)}(b,a) = \bigl(d^B_n b - f_{n-1}(a),\,-d^A_{n - 1} a\bigr).
	\]
	One checks that \( d^{C(f)} \circ d^{C(f)} = 0 \).
\end{definition*}

\begin{remark*}
	The \emdef{shift functor} \( [1]: \Ch(\ct{A}) \to \Ch(\ct{A}) \) is defined by
	\[
		(A[1])_n := A_{n-1}, \qquad d^{A[1]}_n := -\,d^A_{n-1},
	\]
	and for a chain map \( f \) set \( \bigl(f[1]\bigr)_n := f_{n-1} \).
\end{remark*}

\begin{proposition*}[Properties of cones]
	\mbox{}
	\begin{enumerate}
		\item There is a short exact sequence of chain complexes
			\[
				0 \too B \tox{i} C(f) \tox{p} A[1] \too 0,
			\]
			where \( i(b) := (b,0) \) and \( p(b,a) := -a \).
		\item If the square
			\[
				\begin{tikzcd}
					A \arrow[r,"f"] \arrow[d,"g"'] & B \arrow[d,"h"] \\
					A' \arrow[r,"f'"'] & B'
				\end{tikzcd}
			\]
			is a commutative square of chain complexes, then the map
			\[
				h \oplus g[1]: C(f) \too C(f'), \qquad (b,a) \longmapsto \bigl(h(b),\, g(a)\bigr),
			\]
			is a chain map.
		\item In the situation of (2), there is a morphism of short exact sequences
			\[
				\begin{tikzcd}
					0 \arrow[r] & B \arrow[r,"i"] \arrow[d,"h"'] & C(f) \arrow[r,"p"] \arrow[d,"h \oplus g\lbrack1\rbrack"] & A[1] \arrow[r] \arrow[d,"g\lbrack1\rbrack"] & 0 \\
					0 \arrow[r] & B' \arrow[r,"i'"] & C(f') \arrow[r,"p'"] & A'[1] \arrow[r] & 0
				\end{tikzcd}
			\]
			commuting with the indicated arrows.
	\end{enumerate}
\end{proposition*}

Let \( 0 \to A \tox{i} B \tox{q} C \to 0 \) be a short exact sequence in an abelian category \( \ct{A} \). Take projective resolutions \( \varepsilon^A: Q_\bullet \epi A \) and \( \varepsilon^C: P_\bullet \epi C \). By projectivity choose a lift \( \psi_0: P_0 \to B \) with \( q\,\psi_0 = \varepsilon^C_0 \).

Using the induced map on cycles, pick \( \varphi_0: P_1 \to Q_0 \) so that
\[
	i\, \varepsilon^A_0\, \varphi_0 \;=\; \psi_0\, d^{P}_1.
\]
Inductively, choose maps \( \varphi_n: P_{n+1} \to Q_n \) making \( i\, \varepsilon^A_n\, \varphi_n = \psi_n\, d^{P}_{n+1} \) with suitable lifts \( \psi_n: P_n \to B \) extending \( \psi_{n-1} \). This yields a chain map
\[
	\varphi_\bullet: P_\bullet \too Q_\bullet[1].
\]

\begin{lemma*}[Horseshoe Lemma]
	\mbox{}
	\begin{enumerate}
		\item Put \( S_\bullet := C\bigl(\varphi_\bullet[-1]\bigr) \). Then there is a short exact sequence of complexes
			\[
				0 \too Q_\bullet \too S_\bullet \too P_\bullet \too 0,
			\]
			and \( S_\bullet \) is a projective resolution of \( B \) fitting into a commutative diagram with \( \varepsilon^A,\varepsilon^B,\varepsilon^C \).
		\item The construction in (1) is natural with respect to morphisms of short exact sequences.
	\end{enumerate}
\end{lemma*}

We omit the proof of this lemma but reformulate the statement (2):

\begin{lemma*}
	Given a morphism of short exact sequences
	\[
		\begin{tikzcd}
			0 \arrow[r] & A \arrow[r,hook,"i"] \arrow[d,"a"'] & B \arrow[r,two heads,"q"] \arrow[d,"b"] & C \arrow[r] \arrow[d,"c"] & 0 \\
			0 \arrow[r] & A' \arrow[r,hook,"i'"] & B' \arrow[r,two heads,"q'"] & C' \arrow[r] & 0
		\end{tikzcd}
	\]
	choose projective resolutions \( P_\bullet(A), P_\bullet(C) \) and \( P_\bullet(A'), P_\bullet(C') \), together with chain maps lifting \( a \) and \( c \). Then the above construction yields a chain map
	\[
		P_\bullet(B) \too P_\bullet(B'),
	\]
	which makes the diagram of short exact sequences of complexes
	\[
		\begin{tikzcd}
			0 \arrow[r] & P_\bullet(A) \arrow[r] \arrow[d] & P_\bullet(B) \arrow[r] \arrow[d] & P_\bullet(C) \arrow[r] \arrow[d] & 0 \\
			0 \arrow[r] & P_\bullet(A') \arrow[r] & P_\bullet(B') \arrow[r] & P_\bullet(C') \arrow[r] & 0
		\end{tikzcd}
	\]
	commute.
\end{lemma*}

\section{Derived Functors}

\begin{definition*}
	Let \( F: \ct{A} \to \ct{B} \) be a right exact additive functor between abelian categories and assume \( \ct{A} \) has enough projectives. For \( A \in \Ob(\ct{A}) \), choose a projective resolution \( P_\bullet \epi A \) and define
	\[
		(\mrm{L}_i F)(A) := H_i\bigl(F(P_\bullet)\bigr) \qquad (i \ge 0).
	\]
	Independent of the choice of \( P_\bullet \) up to canonical isomorphism, this yields \emdef{left derived functors}
	\[
		\mrm{L}_i F: \ct{A} \to \ct{B}.
	\]
\end{definition*}

\begin{definition*}
	Similarly, if \( F: \ct{A} \to \ct{B} \) is left exact additive and \( \ct{A} \) has enough injectives, then for \( A \in \Ob(\ct{A}) \) and an injective resolution \( A \inc E^\bullet \) one defines \emdef{right derived functors}
	\[
		(\mrm{R}^i F)(A) := H^i\bigl(F(E^\bullet)\bigr) \qquad (i \ge 0).
	\]
\end{definition*}

\begin{proposition*}[Properties of derived functors]
	\mbox{}
	\begin{enumerate}
		\item \( \mrm{L}_0 F \cong F \).
		\item The functors \( \mrm{L}_i F \) are independent of the chosen projective resolution.
		\item Each \( \mrm{L}_i F \) is additive.
	\end{enumerate}
\end{proposition*}
\begin{proof}
	\mbox{}
	\begin{enumerate}
		\item If \( 0 \to P_1 \to P_0 \epi A \to 0 \) is exact with \( P_j \) projective, then applying \( F \) gives an exact sequence \( F(P_1) \to F(P_0) \to F(A) \to 0 \), and hence \( H_0(F(P_\bullet)) \cong F(A) \).
		\item Any two resolutions of \( A \) are connected by a chain homotopy equivalence, and applying \( F \) preserves homotopies, so the induced maps on homology are isomorphisms.
		\item For a morphism \( f: A \to A' \) and chosen projective resolutions, any lift \( f_\bullet \)  of \( f \) induces a chain map whose effect on homology defines \( \mrm{L}_i F(f) \). In particular, \( f_\bullet + g_\bullet \) lifts \( f + g \).
	\end{enumerate}
\end{proof}

\begin{remark*}
	For any projective object \( P \) and any \( i\ge 1 \) one has \( L_iF(P)=0 \).
\end{remark*}

In the category of \( R \)-modules there are two canonical examples.

\begin{examples*}
	\item Given \( M \in \Ob(\cthyprm{R}{Mod}) \), the functor \( h_M := \Hom(M, -) \) is left exact. Then
		\[
			\Ext_R^i(M, N) := R^i \, h_M(N).
		\]
	\item Given \( M \in \Ob(\cthyprm{R}{Mod}) \), the functor \( t_M := M \otimes - \) is left exact. Then
		\[
			\Tor^R_i(M, N) := L^i \, t_M(N).
		\]
\end{examples*}

\begin{theorem*}[Derived long exact sequence]
	Let \( 0 \to A' \tox{i} A \tox{q} A'' \to 0 \) be a short exact sequence in an abelian category \( \ct{A} \), and let \( F: \ct{A} \to \ct{B} \) be a right exact additive functor. Assume \( \ct{A} \) has enough projectives. Then there is a natural long exact sequence
	\[
		\dotsm \too \mrm{L}_{n}F(A') \tox{u_n} \mrm{L}_{n}F(A) \tox{v_n} \mrm{L}_{n}F(A'') \tox{\partial_n} \mrm{L}_{n-1}F(A') \too
	\]
	\[
		\dotsm \too \mrm{L}_{1}F(A'') \tox{\partial_1} \mrm{L}_{0}F(A') \too \mrm{L}_{0}F(A) \too \mrm{L}_{0}F(A'') \to 0.
	\]
	The maps \( u_n \) and \( v_n \) are induced by \( i \) and \( q \), and \( \partial_n \) is the connecting morphism. The construction is functorial in morphisms of short exact sequences.
\end{theorem*}

\begin{proof}
	We outline a proof sketch. Choose projective resolutions \( P'_\bullet \epi A' \) and \( P''_\bullet \epi A'' \). By the Horseshoe Lemma there exists a short exact sequence of complexes
	\[
		0 \too P'_\bullet \too P_\bullet \too P''_\bullet \too 0,
	\]
	with degreewise splittings \( P_n \cong P'_n \oplus P''_n \). Applying \( F \) yields a short exact sequence of complexes
	\[
		0 \too F(P'_\bullet) \too F(P_\bullet) \too F(P''_\bullet) \too 0,
	\]
	so passing to homology gives the desired long exact sequence in which \( H_n(F(P_\bullet)) = \mrm{L}_nF(-) \). Naturality follows from the functoriality of the horseshoe and the connecting morphism in homology.
\end{proof}

\begin{example*}
	Let \( 0 \to A' \tox{i} A \tox{q} A'' \to 0 \) be a short exact sequence of left \( R \)-modules and \( M \) a right \( R \)-module. Then there is a long exact sequence
	\[
		\dotsc \to \Tor_1^R(M, A') \to \Tor_1^R(M, A) \to \Tor_1^R(M, A'') \to M \otimes A' \to M \otimes A \to M \otimes M'' \to 0.
	\]
\end{example*}
