\Lecture{March 11, 2025}

\section{Coequalizers}

Let \( I \) be an indexing category consisting of two parallel arrows. A functor
\[
	D: I \to \C
\]
thereby defines a diagram
\[
	\begin{tikzcd}
		X &&& Y
		\arrow["f", shift left, from=1-1, to=1-4]
		\arrow["g"', shift right, from=1-1, to=1-4]
	\end{tikzcd}
\]
The colimit of this diagram is the \emdef{coequalizer} of \( f \) and \( g \):
\[
	\begin{tikzcd}
		X && Y && {\coeq(f,g)} \\
		\\
		&&&& T
		\arrow["f", shift left, from=1-1, to=1-3]
		\arrow["g"', shift right, from=1-1, to=1-3]
		\arrow["q"{description}, from=1-3, to=1-5]
		\arrow["t"', from=1-3, to=3-5]
		\arrow["{\exists! u}", dashed, from=1-5, to=3-5]
	\end{tikzcd}
\]
This diagram is called a \emdef{cofork}.

\begin{examples*}
	\item A special case in \( \ctrm{Sets} \). Let \( R \subset X \times X \) be an equivalence relation on \( X \).
		\[
			\begin{tikzcd}
				R && {X \times X} && X && R && X
				\arrow["r", hook, from=1-1, to=1-3]
				\arrow["{\pi_1}", shift left, from=1-3, to=1-5]
				\arrow["{\pi_2}"', shift right, from=1-3, to=1-5]
				\arrow["{r_1}", shift left, from=1-7, to=1-9]
				\arrow["{r_2}"', shift right, from=1-7, to=1-9]
			\end{tikzcd}
		\]
		\[
			\begin{tikzcd}
				R && X && {X/R} \\
				\\
				&&&& Z
				\arrow["{r_1}", shift left, from=1-1, to=1-3]
				\arrow["{r_2}"', shift right, from=1-1, to=1-3]
				\arrow["q", from=1-3, to=1-5]
				\arrow["h"', from=1-3, to=3-5]
				\arrow["{\exists! u}", dashed, from=1-5, to=3-5]
			\end{tikzcd}
		\]
	\item In general, the coequalizer of \( f \) and \( g \) is
		\[
			\coeq(f, g) = Y/R^E,
		\]
		where \( R^E \) is the equivalence relation generated by
		\[
			T = \{(f(x), g(x)) \in Y \times Y \,\mid\, x \in X\}.
		\]
\end{examples*}

\section{Pushouts}

Let \( I \) be the category \( (\bullet \ot \bullet \to \bullet) \) and \( D: I \to \C \). The colimit of \( D \)
\[
	\begin{tikzcd}
		Z && Y \\
		\\
		X && {X \sqcup_Z Y} \\
		\\
		&&&& T
		\arrow["g", from=1-1, to=1-3]
		\arrow["f"', from=1-1, to=3-1]
		\arrow["{q_2}"', from=1-3, to=3-3]
		\arrow["{t_2}", from=1-3, to=5-5]
		\arrow["{q_1}", from=3-1, to=3-3]
		\arrow["{t_1}"', from=3-1, to=5-5]
		\arrow["{\exists!u}"{description}, dashed, from=3-3, to=5-5]
	\end{tikzcd}
\]
is called the \emdef{pushout} of \( f \) and \( g \).

\begin{examples*}
	\item In \( \ctrm{Sets} \), the pushout of \( f \) and \( g \) is
		\[
			X \sqcup_Z Y := X \sqcup Y / R,
		\]
		where \( R \) is the equivalence relation on \( X \sqcup Y \) generated by
		\[
			f(z) \sim g(z) \quad \text{ for all } z \in Z.
		\]
		In particular, if \( X, Y \) are subobjects of \( A \),
		\[
			\begin{tikzcd}{X \cap Y} && Y \\
				\\
				X && {X \cup Y}
				\arrow[from=1-1, to=1-3]
				\arrow[from=1-1, to=3-1]
				\arrow[from=1-3, to=3-3]
				\arrow[from=3-1, to=3-3]
			\end{tikzcd}
		\]
	\item In \( \ctrm{Top} \), the pushout is
		\[
			X \sqcup_Z Y := X \sqcup Y \fac R
		\]
		with the quotient topology.
		\[
			\begin{tikzcd}{S^1} && {D^2} \\
				\\
				{D^2} && {S^2}
				\arrow[hook, from=1-1, to=1-3]
				\arrow[hook', from=1-1, to=3-1]
				\arrow[from=1-3, to=3-3]
				\arrow[from=3-1, to=3-3]
			\end{tikzcd}
		\]
	\item In \( \ctrm{Grps} \), given the diagram
		\[
			G \otx{f} K \tox{g} H
		\]
		the pushout of \( f \) and \( g \) is the \emdef{amalgamated product}
		\[
			G \ast_K H = G \ast H \fac N,
		\]
		where \( N \) is the normailzation of \( \{f(k) g(k)^{-1}\}_{k \in K} \).

		\vspace*{3mm}

		Let \( \bb{Z}/4\bb{Z} = \cl{S} \) and \( \bb{Z}/6\bb{Z} = \cl{T} \),
		\[
			\bb{Z}/4\bb{Z} \ast_{\bb{Z}/2\bb{Z}} \bb{Z}/6\bb{Z} = \cl{S, T ~|~ S^4, T^6, S^2 = T^3} \cong SL_2(\bb{Z}),
		\]
		where the last isomorphism is defined by
		\[
			S \mapsto
			\begin{pmatrix}
				0 & -1 \\
				1 & 0
			\end{pmatrix}
			, \quad T \mapsto
			\begin{pmatrix}
				1 & -1 \\
				1 & 0
			\end{pmatrix}
			.
		\]
		\begin{remark*}
			\( PSL_2(\bb{Z}) = SL_2(\bb{Z}) / (t \Id) = \cong \bb{Z}/2\bb{Z} \ast \bb{Z}/3\bb{Z} \).
		\end{remark*}

	\item Seifert-Van Kampan Theorem:

		\vspace*{2mm}

		If a topological space \( X = U_1 \cup U_2 \), \( U_1, U_2, U_1 \cap U_2 \) are path-connected, then
		\[
			\begin{tikzcd}{U_1\cap U_2} && {U_2} && {\pi_1(U_1\cap U_2, x)} && {\pi_1(U_2, x)} \\
				\\
				{U_1} && X && {\pi_1(U_1, x)} && {\pi_1(X, x)}
				\arrow[from=1-1, to=1-3]
				\arrow[from=1-1, to=3-1]
				\arrow[from=1-3, to=3-3]
				\arrow[from=1-5, to=1-7]
				\arrow[from=1-5, to=3-5]
				\arrow[from=1-7, to=3-7]
				\arrow[from=3-1, to=3-3]
				\arrow[from=3-5, to=3-7]
			\end{tikzcd}
		\]
		the last diagram is a pushout square for any \( x \in U_1 \cap U_2 \).

		\vspace*{2mm}

		So the fundamental group \enquote{maps pushouts to pushouts}.

	\item In \( \ctrm{Rings} \), the pushout is the \emdef{amalgamated product of rings}.

	\item In \( \ctrm{CRings} \), the pushout is the tensor product of algebras.
\end{examples*}

\section{Direct limits}

\begin{definition*}
	A \emdef{direct limit} is a colimit over a direct set.
\end{definition*}

\begin{examples*}
	\item In \( \ctrm{Sets} \), the direct limit of
		\[
			\begin{tikzcd}{X_0} && {X_1} && {X_2} && \dotsc
				\arrow[hook, from=1-1, to=1-3]
				\arrow[hook, from=1-3, to=1-5]
				\arrow[hook, from=1-5, to=1-7]
			\end{tikzcd}
		\]
		is the union
		\[
			\liminj X_n = \bigcup\limits_{n=0}^\infty X_n.
		\]
	\item In \( \ctrm{Grps} \), the direct limit of
		\[
			\begin{tikzcd}{S_1} && {S_2} && {S_2} && \dotsc
				\arrow[hook, from=1-1, to=1-3]
				\arrow[hook, from=1-3, to=1-5]
				\arrow[hook, from=1-5, to=1-7]
			\end{tikzcd}
		\]
		is the group
		\[
			\liminj S_n = \{\sigma \in S_{\bb{N}} ~|~ \sigma(n) = n \text{ for almost all } n\}.
		\]
		Given a ring \( R \), the direct limit of
		\[
			\begin{tikzcd}{GL_1(R)} && {GL_2(R)} && {GL_3(R)} && \dotsc
				\arrow[hook, from=1-1, to=1-3]
				\arrow[hook, from=1-3, to=1-5]
				\arrow[hook, from=1-5, to=1-7]
			\end{tikzcd}
		\]
		is the \emdef{quite general group}
		\[
			\liminj GL_n(R) = GL(R).
		\]
	\item In \( \ctrm{Ab} \), the direct limit of
		\[
			\begin{tikzcd}{\bb{Z}/p\bb{Z}} && {\bb{Z}/p^2\bb{Z}} && {\bb{Z}/p^3\bb{Z}} && \dotsc
				\arrow["p", hook, from=1-1, to=1-3]
				\arrow["p", hook, from=1-3, to=1-5]
				\arrow["p", hook, from=1-5, to=1-7]
			\end{tikzcd}
		\]
		is the p-\emdef{Prufer group}
		\[
			\liminj \bb{Z}/p^n\bb{Z} = \bb{Z}(p^\infty) = \bb{Z}[1/p] \fac \bb{Z}.
		\]
\end{examples*}

\section{Functors and Limits}

\begin{definition*}
	A functor \( F: \C \to \ct{D} \) \emdef{preserves limits} if for any \( D: I \to \C \) and its limiting cone
	\[
		(L \tox{p_i} D(i))_{i \in I},
	\]
	the cone
	\[
		(F(L) \tox{F(p_i)} FD(i))_{i \in I}
	\]
	is limiting for \( FD: I \to \ct{D} \).
\end{definition*}

\begin{theorem*}
	The functor \( h_X \) preserves limits:
	\[
		\C(X, \lim_I D) = \lim_I \C(X, D).
	\]
	The isomorphism is natural in \( X \) and \( D \).
\end{theorem*}
\begin{proof}
	Let \( L \tox{p_i} D(i) \) and \( Z \) be a set, then
	\[
		\begin{tikzcd}
			L && {D(i)} && {\C(X, L)} && {\C(X, D(i))} \\
			\\
			X &&&& {z \in Z}
			\arrow["{p_1}", from=1-1, to=1-3]
			\arrow["{\C(X, p_i)}"', from=1-5, to=1-7]
			\arrow["{u(z)}", dashed, from=3-1, to=1-1]
			\arrow["{q_i(z)}"', from=3-1, to=1-3]
			\arrow[dashed, from=3-5, to=1-5]
			\arrow["{q_i}"', from=3-5, to=1-7]
		\end{tikzcd}
	\]
\end{proof}

\begin{corollary*}
	\( h^X(\colim D) = \lim h^X(D) \).
\end{corollary*}

\begin{corollary*}
	Let \( F: \C \to \ct{D} \) and \( G: \ct{D} \to \C \) such that \( F \adj G \), then
	\begin{enumerate}
		\item \( F \) preserves colimits (LAPC);
		\item \( G \) preserves limits (RAPL).
	\end{enumerate}
\end{corollary*}
\begin{proof}
	We show (2):
	\[
		\C(C, G(\lim_I D)) \cong \ct{D}(F(C), \lim_I D) \cong \lim_I \ct{D}(F(C), D(i)) \cong
	\]
	\[
		\cong \lim_I \C(C, GD(i)) \cong \C(C, \lim_I GD(i)).
	\]
	By Yoneda's lemma, it follows that
	\[
		G(\lim_I D) \cong \lim_I GD.
	\]
\end{proof}

\begin{examples*}
	\item Let \( R \) and \( S \) be unital associative rings, and \( _R B_S \) be an \( R\text{-}S \)-bimodule.
		\[
			\begin{tikzcd}{\cthyprm{S}{Mod}} &&& {\cthyprm{R}{Mod}}
				\arrow[""{name=0, anchor=center, inner sep=0}, "F", shift left=2, from=1-1, to=1-4]
				\arrow[""{name=1, anchor=center, inner sep=0}, "G", shift left=2, from=1-4, to=1-1]
				\arrow["\adj"{anchor=center, rotate=-90}, draw=none, from=0, to=1]
			\end{tikzcd}
		\]
		And functors \( F \) and \( G \) are defined by
		\[
			F(M) = B \otimes_S M, \quad G(N) = \Hom_R(B, N).
		\]
		The required isomorphism is
		\[
			\Hom_R(B \otimes_S M, N) \cong \Hom_S(M, \Hom_R(B, N)),
		\]
		where
		\[
			f \,\mapsto\, (m \mapsto f((-) \otimes m))
		\]
		and
		\[
			(b \otimes m \mapsto g(m) b) \,\mapsot\, g.
		\]
		Since \( B \otimes - \) preserves colimits:
		\begin{enumerate}
			\item \( B \otimes (\bigoplus M_i) \cong \bigoplus (B \otimes M_i) \)
			\item it is right exact (=preserves cokernels), i.e. exactness of
				\[
					M' \tox{f} M \to M'' \to 0
				\]
				implies the exactness of
				\[
					B \otimes M' \to B \otimes M \to B \otimes M'' \to 0.
				\]
		\end{enumerate}
\end{examples*}

\begin{remark*}
	Let \( f: R \to S \) be a morhisms of rings, the \emdef{restriction of scalars} functor
	\[
		f^* \equiv \opn{res}_f: \cthyprm{S}{Mod} \to \cthyprm{R}{Mod}
	\]
	on an \( S \)-module \( N \) is simultaneously
	\[
		f^*(N) \cong \Hom_S(_S S_R, N) = _R S_S \otimes_S N.
	\]
	This gives us adjunctions
	\[
		f_! \equiv \opn{ind} \adj f^* \adj f_* \equiv \opn{coind},
	\]
	where
	\[
		f_!(M) = _S S_R \otimes_R M, \quad f_*(M) = \Hom_R(_R S_S, M).
	\]
\end{remark*}