\Lecture{March 18, 2025}

\textsc{In the last episode...}

\vspace*{2mm}

Let \( F: \C \to \ct{D} \) and \( G: \ct{D} \to \C \) be a pair of adjoint functors (\( F \adj G \)).

\begin{definition*}
	A functor is called \emdef{continuous} (resp. \emdef{cocontinuous}) if it preserves all limits (resp. colimits).
\end{definition*}

We note that \( F \) is cocontinuous (LAPC) and \( G \) is continuous (RAPL).

\begin{examples*}
	\item Let \( R, S \) be rings, and \( _R B_S \) an \( R \)-\( S \)-bimodule. Then
		\[
			B_S \otimes_S (-) \,\adj\, \Hom_R(_R B, -),
		\]
		hence in particular
		\[
			B_S \otimes_S \colim N_i \cong \colim B_S \otimes_S N_i
		\]
		and
		\[
			\Hom_R(_R B, \lim M_i) \cong \lim \Hom_R (_R B, M_i).
		\]
	\item Since \( \colim_I \adj \Delta \adj \lim_I \), we get (completely for free)
		\[
			\lim_I \lim_J(-) \cong \lim_J \lim_I (-).
		\]
\end{examples*}

\chapter{Adjoint Functor Theorems}

\section{Generators and Cogenerators of Categories}

\begin{definitions*}
	\item A set of objects \( \{G_i\}_{i \in I} \) in a category \( \C \) is a set of \emdef{generators} (\emdef{separators}) of \( \C \) if for all \( f \neq g \in \C(X, Y) \) there exist \( i \in I \) and \( h: G_i \to X \) such that \( fh \neq gh \).
	\item Object \( G \) is called a generator if \( \{G\} \) is a generating set.
	\item A \emdef{cogenerator} is a generator in the opposite category.
\end{definitions*}

\begin{remarks*}
	\item \( G \) is a generator if and only if \( h_G = \C(G, -) \) is faithful.
	\item \( \{G_i\}_{i \in I} \) is a generating set if and only if \( \prod_{i \in I} h_{G_i} \) is faithful.
\end{remarks*}

\begin{examples*}
	\item In \( \ctrm{Sets} \), any non-empty set is a generator. Moreover, any \( S \) with \( |S| \ge 2 \) is a cogenerator.
	\item In \( \ctrm{Top} \) we can endow the corresponding sets with discrete (resp. indiscrete) topology.
	\item In \( \ctrm{Ab} \), a group \( \bb{Z} \oplus A \) is a generator and \( \bb{Q}/\bb{Z} \) is a cogenerator.
	\item In \( \cthyprm{R}{Mod} \), \( R \) is a generator and \( \Hom_\bb{Z}(_\bb{Z} R, \bb{Q}/\bb{Z}) \) is a cogenerator.
	\item In \( \ctrm{Grps} \), \( \bb{Z} \) is a generator.
		\begin{statement*}
			There are no cogenerators in \( \ctrm{Grps} \).
		\end{statement*}
		\begin{proof}
			Suppose \( Q \) is cogenerating. Given a non-trivial simple group \( G \),
			\[
				\id_G \neq 1: G \to G
			\]
			hence there is \( \alpha: G \to Q \) such that \( \alpha \neq 1 \). Since \( \ker \alpha \) is a proper normal subgroup of a simple group, \( \alpha \) is injective. But there are simple groups of arbitrary large cardinalities (for example, \( \mrm{PSL}_2(K), K = \bb{C}(x_i)_{i \in I} \)), a contradiction.
		\end{proof}
	\item In \( \ctrm{Rings} \), \( \bb{Z}[x] \) is a generator. There are no cogenerators by the similar agrument: there are fields of arbitrary large cardinalities, and all non-trivial homomorphisms from them are injective.
	\item In \( \ctrm{CHaus} \), \( \{\ast\} \) is a generator, and \( [0, 1] \) is a cogenerator.
\end{examples*}

\begin{definitions*}
	\item Let \( R \incx{f} C \) and \( S \incx{g} C \) be monomorphisms in \( \C \). We say that \( (R, f) \sim (S, g) \) if there is an isomorphism \( \tau: R \to S \) such that \( g \tau = f \). Equivalence classes under this relation are called \emdef{subobjects} of \( C \). We denote by \( \mrm{Sub}(C) \) the class of all subobjects of \( C \).
	\item We say that \( \C \) is \emdef{well-powered} if \( \mrm{Sub}(C) \) is a set for all \( C \in \Ob(\C) \).
\end{definitions*}

\begin{examples*}
	\item All \enquote{everyday} categories are well-powered.
	\item An example of a not well-powered category would be any partially ordered class with a maximal element (which every element would be a distinct suboject of). In particular, \( \ctrm{Ord}^\op \).
\end{examples*}

\begin{theorem*}[Electrification]
	Let \( \C \) be a balanced category with finite intersections and a set of generators \( \{G_i\} \). Then \( \C \) is well-powered.
\end{theorem*}
\begin{proof}
	Let \( B \not\cong B' \). If both \( j \) and \( j' \) are epic, then \( B \cong B \cap B' \cong B' \), a contradiction.
	\[
		\begin{tikzcd}{G_i} && {B \cap B'} && {B'} \\
			\\
			&& B && C
			\arrow["{\nexists \tilde{h}}", from=1-1, to=1-3]
			\arrow["b"', from=1-1, to=3-3]
			\arrow["j", hook, from=1-3, to=1-5]
			\arrow["{j'}"', hook, from=1-3, to=3-3]
			\arrow["{i'}", hook, from=1-5, to=3-5]
			\arrow["i"', hook, from=3-3, to=3-5]
		\end{tikzcd}
	\]

	Suppose (WLOG) \( j \) is not epic, i.e. there are \( f \neq g: B \to D \) such that \( fj = gj \).

	\vspace*{2mm}

	There is \( h: G_i \to B \) such that \( fh \neq gh \). It is clear that \( h \) cannot be factored through \( B \cap B' \). Hence, there is no \( h': G_i \to B' \) such that \( i' h' = ih \). It implies that
	\[
		\Phi: \opn{Sub}(\C) \to 2^{\C(G_i, C)}, \quad B \mapsto \C(G_i, B)
	\]
	is injective.
\end{proof}

\begin{lemma*}
	Let \( \C \) and \( \ct{D} \) be complete categories and \( F: \C \to \ct{D} \), \( G: \ct{D} \to \ct{E} \) continuous functors. Then \( F \comma G \) is complete, and the forgetful functors \( P_{\C}: F \comma G \to \C \) and \( P_{\ct{D}}: F \comma G \to \ct{D} \) are continuous.
\end{lemma*}
\begin{proof}
	We provide a sketch of the proof:
	\[
		\lim_I\bigl(F(C_i) \tox{f} G(C_i)\bigr) = \lim_I F(C_i) \tox{\lim f_i} G(C_i).
	\]
\end{proof}

\begin{theorem*}[Primeval AFT]
	Suppose \( \ct{D} \) has all (not necessary small) limits. A functor \( G: \ct{D} \to \C \) has a left adjoint if and only if \( G \) is continuous.
\end{theorem*}

\begin{remark*}
	For such \( \ct{D} \) we have \( |\ct{D}(X, Y)| \le 1 \) (i.e. \( \ct{D} \) is \emdef{thin}).
\end{remark*}
\begin{proof}
	One direction follows from RAPL. Now, by the previous lemma, \( \C \comma G \) has all limits, hence \( \lim \Id_{\C \comma G} \) is the initial object.
\end{proof}

\begin{lemma*}
	If \( \ct{D} \) is complete, well-powered with a cogenerator \( Q \), then \( \ct{D} \) has an initial object.
\end{lemma*}
\begin{proof}
	Let \( i = \bigcap\limits_{X \inc Q} X \) and \( D \in \Ob(\ct{D}) \). For \( f, g \in \ct{D}(i, D) \), their equalizer is a suboject of \( i \), hence \( f = g \). Set \( S := \ct{D}(D, Q) \). The map \( D \to Q^S \) is monic, therefore
	\[
		\begin{tikzcd}
			T && Q \\
			\\
			D && {Q^S}
			\arrow[hook, from=1-1, to=1-3]
			\arrow[from=1-1, to=3-1]
			\arrow["\Delta", from=1-3, to=3-3]
			\arrow[hook, from=3-1, to=3-3]
		\end{tikzcd}
	\]
	is a pullback square. Now \( i \to T \to D \) is a required map.
\end{proof}

\begin{proposition*}
	Let \( \C \) be a category with (small) coproducts and \( \{G_I\}_{i \in I} \) be a set of objects. Then the following are equivalent:
	\begin{enumerate}
		\item \( \{G_i\}_{i \in I} \) is a generating set.
		\item \( G = \coprod G_i \) is a generator.
		\item For each \( X \in \Ob(\C) \) there is a set \( S \) and an epimorphism \( G^{(S)} \to X \).
	\end{enumerate}
\end{proposition*}

\begin{theorem*}[SAFT]
	Let \( \ct{D} \) be a complete well-powered category with a cogenerator \( Q \). A functor \( G: \ct{D} \to \C \) has a left adjoint if and only if \( G \) is continuous.
\end{theorem*}
\begin{proof}
	Only one direction requires proof. We want to show that \( C \comma G \) has an initial object for each \( C \in \Ob(\C) \). To do that we apply the previous lemma:
	\begin{enumerate}
		\item \( C \comma G \) is complete (DONE).
		\item \( C \comma G \) is well-powered:

			Since \( C \comma G \tox{P_{\ct{D}}} \ct{D} \) is continuous, it preserves monomorphisms (since \( f \) is mono iff the corresponding square is a pullback).

			\vspace*{2mm}

			We show that \( P_{\ct{D}} \) reflects monomorphisms. Indeed,
			\[
				\begin{tikzcd}
					&&& {G(D'')} \\
					\\
					C &&& {G(D)} \\
					\\
					&&& {G(D')}
					\arrow["{G(k)}"', shift right=2, from=1-4, to=3-4]
					\arrow["{G(k')}", shift left=2, from=1-4, to=3-4]
					\arrow["{f''}", from=3-1, to=1-4]
					\arrow["f", from=3-1, to=3-4]
					\arrow["{f'}"', from=3-1, to=5-4]
					\arrow["{G(h)}", from=3-4, to=5-4]
				\end{tikzcd}
			\]
			If \( h \) is mono, \( G(h) G(k) = G(h) G(k') \), we get \( G(k) = G(k') \) since \( G(h) \) is monic.
		\item \( C \comma G \) has a cogenerator:

			\vspace*{2mm}

			We show that \( S = \C(C, G(D)) \) is a generating set. Let \( C \tox{f} G(D) \in \Ob(C \comma G) \). We have \( D \inc Q^T \), so the diagram
			\[
				\begin{tikzcd}
					&&& {G(D)} \\
					C \\
					&&& {G(Q)^T}
					\arrow["{G(j)}", from=1-4, to=3-4]
					\arrow["f", from=2-1, to=1-4]
					\arrow["{G(j) f}"', from=2-1, to=3-4]
				\end{tikzcd}
			\]
			finishes the proof.
	\end{enumerate}

\end{proof}