\Lecture{January 21, 2025}

\Heading{Course Outline}
\begin{enumerate}
	\item Categories, functors, natural transformations
	\item Adjoint functors
	\item Limits and colimits
	\item Abelian categories
	\item Resolutions and derived functors
\end{enumerate}

\Heading{Course Grading}
\begin{enumerate}
	\item 3 quizzes: 10\% each, 30\% total
	\item Midterm exam: 30\%
	\item Final exam: 30\%
	\item Seminars: 10\%
	\item Homework: \( {\sim}10\% \)
\end{enumerate}

\chapter{Categories and Functors}

\section{Definition and Examples of Categories}

\begin{definition*}
	A \emdef{category} \( \C \) consists of the following data:
	\begin{enumerate}[(i)]
		\item a class of \emdef{objects} \( \Ob(\C) \);
		\item for every \( X, Y \in \Ob(\C) \) a set (or a class) of \emdef{morphisms} \( \Hom_{\C}(X, Y) \);
		\item for every \( X \in \Ob(\C) \) an \emdef{identity morphism} \( \id_X \equiv 1_X \in \Hom_{\C}(X, X) \);
		\item for every \( X, Y, Z \in \Ob(\C) \) a \emdef{composition rule}
			\[
				\circ: \Hom_{\C}(X, Y) \times \Hom_{\C}(Y, Z) \to \Hom_{\C}(X, Z)
			\]
			satisfying the usual associativity and unitality relations.
	\end{enumerate}
	A category \( \C \) is called \emdef{locally small} if each hom-set \( \Hom_{\C}(X, Y) \) is a set.
\end{definition*}

\begin{remark*}
	There are plenty of ways to denote the set of morphisms. For example,
	\[
		\Hom_{\C}(X, Y) \equiv \C(X, Y) \equiv \opn{Maps}_{\C}(X, Y) \equiv \opn{Arr}_{\C}(X, Y).
	\]
\end{remark*}

\begin{definition*}
	Given a category \( \C \), there is an \emdef{opposite category} \( \C^\op \), whose
	\begin{enumerate}[(i)]
		\item objects are the same as in \( \C \), i.e. \( \Ob(\C^\op) := \Ob(\C) \);
		\item morphisms are \enquote{reversed}, i.e. for all \( X, Y \in \Ob(\C^\op) \)
			\[
				\Hom_{\C^\op}(X, Y) := \Hom_{\C}(Y, X)
			\]
			with the natural composition rule \( f \circ_\op g := g \circ f \).
	\end{enumerate}
\end{definition*}

\begin{examples*}
	\item \( \ctrm{Sets} \) is the category of sets and functions;
	\item \( \ctrm{Mon} \) is the category of monoids and monoid homomorphisms;
	\item \( \ctrm{Grps} \) is the category of group and group homomorphisms;
	\item \( \ctrm{Rings} \) is the category of (all) rings and ring homomorphisms;
	\item \( \ctrm{CRings} \) is the category of commutative rings;
	\item \( \ctrm{Fields} \) is the category of fields and field homomorphisms;
	\item For a monoid \( M \), there is a category \( \cthyprm{M}{Sets} \), where
		\begin{enumerate}[(i)]
			\item objects are sets \( X \) with an \( M \)-\emdef{action} \( M \times X \to X \);
			\item morphisms are \emdef{equivariant maps}, i.e. functions \( \varphi: X \to Y \) such that
				\[
					\varphi(mx) = m \varphi(x)
				\]
				for every \( m \in M \) and \( x \in X \). This can be expressed as a \emdef{commutative diagram}
				\[
					\begin{tikzcd}{M \times X} &&& {M \times Y} \\
						\\
						X &&& Y
						\arrow["{1_M \times \varphi}", from=1-1, to=1-4]
						\arrow[from=1-1, to=3-1]
						\arrow[from=1-4, to=3-4]
						\arrow["\varphi"', from=3-1, to=3-4]
					\end{tikzcd}
				\]
				meaning that tracing elements through all possible paths yields the same result:
				\[
					\begin{tikzcd}{(m, x)} &&& {(m, \varphi(x))} \\
						\\
						mx &&& {\varphi(mx) = m \varphi(x)}
						\arrow[maps to, from=1-1, to=1-4]
						\arrow[maps to, from=1-1, to=3-1]
						\arrow[maps to, from=1-4, to=3-4]
						\arrow[from=3-1, to=3-4]
					\end{tikzcd}
				\]
		\end{enumerate}

	\item \( \cthyprm{R}{Mod} \) and \( \ctrmhyp{Mod}{R} \) are the categories of left and right modules over a ring \( R \).

		If \( R \) is commutative, these notions coincide and we write
		\[
			\cthyprm{R}{Mod} \equiv \ctrmhyp{Mod}{R} \equiv \ctrm{Mod}(R).
		\]
		Moreover, if \( R \) is a field, this is the category of vector spaces
		\[
			\ctrm{Mod}(R) \equiv \ctrm{Vect}(R).
		\]

	\item \( \ctrm{Top} \) is the category of topological spaces and continuous maps.
	\item The category of metric spaces \( \ctrm{Met} \):
		\begin{enumerate}[(i)]
			\item Objects are pairs \( (X, d) \), where \( X \) is a set and \( d: X \times X \to \bb{R} \) a metric.
			\item Morphisms \( (X, d_1) \to (Y, d_2) \) are functions \( f: X \to Y \) which are
				\begin{enumerate}[a)]
					\item isometries (the rigid version of \( \ctrm{Met} \)), i.e.
						\[
							d(f(x), f(y)) = d(x, y) \quad \text{ for all } x, y \in X;
						\]
					\item Lipschitz maps (categorically richer version of \( \ctrm{Met} \)), i.e.
						\[
							\exists K > 0: \quad d(f(x), f(y)) \le K d(x, y) \quad \text{ for all } x, y \in X.
						\]
				\end{enumerate}
		\end{enumerate}

	\item \( C^{\infty} \)-manifolds, analytic manifolds, schemes, \dots
	\item \( \ctrm{hTop} \) is the homotopy category of topological spaces, which
		\begin{enumerate}[(i)]
			\item objects are the same as in \( \ctrm{{Top}} \);
			\item morphisms are homotopy classes of continuous maps.
		\end{enumerate}

		\begin{definition*}[intuitive]
			A category is called \emdef{concrete} if objects are sets with additional structure and hom-sets are maps of sets respecting the additional structure.
		\end{definition*}

		\begin{theorem*}[w/o proof]
			\( \ctrm{hTop} \) is not a concrete category.
		\end{theorem*}

	\item The category of \emdef{binary relations} \( \ctrm{Rel} \):
		\begin{enumerate}[(i)]
			\item Objects of \( \Ob(\ctrm{Rel}) \) are sets.
			\item Morphisms \( X \to Y \) are binary relations \( R \subset X \times Y \) with the usual notion of composition.
		\end{enumerate}

	\item Every monoid \( M \) can be considered a category \( \ct{M} \), where
		\begin{enumerate}[(i)]
			\item there is a single object \( \ast \);
			\item morphisms \( \ast \to \ast \) are labeled by elements of \( M \), where composition is plain multiplication.
		\end{enumerate}

		\begin{definition*}
			A \emdef{groupoid} \( \C \) is a category where every arrow is inverible, which formally means that for every \( f \in \C(X, Y) \) there is \( g \in \C(Y, X) \) such that \( fg = 1_Y \) and \( gf = 1_X \).
		\end{definition*}

		\begin{example*}
			If \( G \) is a group, then \( \ct{G} \) is a groupoid.
		\end{example*}

	\item If \( X \) is a topological space, then \( \Pi_1(X) \), the \emdef{fundamental groupoid} of \( X \):
		\begin{enumerate}[(i)]
			\item Objects of \( \Pi_1(X) \) are points of \( X \).
			\item Morphisms \( x \to y \) are homotopy classes of all paths from \( x \) to \( y \).
		\end{enumerate}
		This is a generalization of the fundamental group of \( X \), since
		\[
			\Hom_{\Pi_1(X)}(x, x) = \pi_1(X, x).
		\]
\end{examples*}

\section{Special Morphisms}

Let \( f \in \C(X, Y) \) be a morphism in \( \C \). We say that

\begin{definitions*}
	\item \( f \) is an \emdef{isomorphism} (iso) if it has an inverse \( f^{-1} \in \C(Y, X) \).
	\item \( f \) is a \emdef{monomorphism} (mono) if \( fg_1 = fg_2 \) implies \( g_1 = g_2 \) for any \( g_1, g_2 \in \C(Z, X) \).
	\item \( f \in \C(X, Y) \) is an \emdef{epimorphism} (epi) if \( f \in \C^\op(Y, X) \) is a monomorphism.
	\item Suppose \( Y \tox{g}  X \tox{f} Y \) with \( f \circ g = 1_Y \). Then \( g \) is called a \emdef{section} (or a \emdef{coretraction}) of \( f \) and \( g \) is called a \emdef{retraction} (or a \emdef{cosection}) of \( g \).
\end{definitions*}

\begin{lemma*}[Basic properties of monomorphisms and epimorphisms]
	\hfill
	\begin{enumerate}
		\item Composition of monomorphisms (resp. epimorphisms) is monic (resp. epic);
		\item If \( fg \) is monic, then \( g \) is monic;
		\item If \( fg \) is epic, then \( f \) is epic;
		\item Any section is a monomorphism;
		\item Any retraction is an epimorphism.
	\end{enumerate}
\end{lemma*}
\begin{proof}
	\textsc{Exercise.}
\end{proof}

\vspace*{2mm}

Note that there are non-invertible maps that are both monomorphisms and epimorphisms.

\vspace*{2mm}

\begin{lemma*}
	For \( f: X \to Y \), the following are equivalent:
	\begin{enumerate}
		\item \( f \) is an isomorphism;
		\item \( f \) is a retraction and a monomorphism;
		\item \( f \) is a section and an epimorphism.
	\end{enumerate}
\end{lemma*}
\begin{proof}
	\textsc{Exercise.}
\end{proof}

\begin{remarks*}
	\item Sections are also called \emdef{split monomorphisms}.
	\item Similarly, retractions are called \emdef{split epimorphisms}.
\end{remarks*}

\begin{definition*}
	A category \( \C' \) is a \emdef{subcategory} of a category \( \C \) if
	\begin{enumerate}[(i)]
		\item its objects form a subclass of objects of \( \C \), i.e. \( \Ob(\C') \subset \Ob(\C) \);
		\item its morphisms are \( \ct{C'}(X, Y) \subset \C(X, Y) \) for all \( X, Y \in \Ob(\ct{C'}) \).
	\end{enumerate}
	A subcategory is called \emdef{full} if \( \ct{C'}(X, Y) = \C(X, Y) \) for each \( X, Y \in \Ob(\ct{C'}) \).
\end{definition*}