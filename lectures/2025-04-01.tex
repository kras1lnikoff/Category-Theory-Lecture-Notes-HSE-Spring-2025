\Lecture{April 1, 2025}

\section{Eilenberg-Watts Theorems}

\begin{enumerate}
	\item Let \( G: \ctRMod \to \ctRMod[S] \) be a continuous functor. By SAFT \( G \) has a left adjoint \( F \). Then for any \( R \)-module \( M \)
		\[
			G(M) = \Hom_S(S, G(M)) \cong \Hom_R(F(S), M).
		\]
		Note that \( F(S) \) is a bimodule, since we have \( S \to \End_S(S) \to \End_R(F(S)) \).
	\item  Let \( F: \ctRMod[S] \to \ctRMod \) be a cocontinuous functor. By co-SAFT it has a right adjoint \( G \). Then \( G = \Hom(B, -) \), hence \( F = B \otimes_S - \).
\end{enumerate}

\begin{definition*}
	A set of objects \( \{D_i\}_{i \in I} \) is called \emdef{weakly initial} if for all \( D \in \Ob(\D) \) there is \( i \in I \) and a map \( D_i \to D \).
\end{definition*}

\begin{lemma*}
	Let \( \D \) be a complete category. If \( \D \) has a weakly initial set, then it has an initial object.
\end{lemma*}
\begin{proof}
	Let \( W := \prod D_i \) and \( V := \eq(\D(W, W)) \inc W \). We know that \( gh = h \) for each \( g \in \D(W, W) \).
	\begin{enumerate}
		\item[1)] Let \( D \in \Ob(\D) \). There is \( D_i \to D \), hence \( V \tox{h} W \tox{p_i} D_i \to D \) and \( \D(V, D) \neq \varnothing \).
		\item[2)] Suppose \( f,g: V \to D \). We want to show that \( f = g \).
			\[
				\begin{tikzcd}
					E && V && D \\
					\\
					W && W
					\arrow["e", hook, from=1-1, to=1-3]
					\arrow["f", shift left, from=1-3, to=1-5]
					\arrow["g"', shift right, from=1-3, to=1-5]
					\arrow["h"', from=1-3, to=3-3]
					\arrow["s", from=3-1, to=1-1]
					\arrow["hes"', from=3-1, to=3-3]
				\end{tikzcd}
			\]
			Since \( hesh = h \) and \( h \) is monic, \( esh = 1_V \). Analogously, \( she = 1_E \) given that \( e \) is monic. But \( e \) was a regular monomorphism, hence \( e \) is an isomorphism. Therefore, \( f = g \).
	\end{enumerate}
\end{proof}

\begin{theorem*}[GAFT or Freyd's AFT]
	Let \( \D \) be a complete category and \( G: \D \to \C \) a continuous functor. Then \( G \) has a left adjoint if and only if for each \( C \in \Ob(\C) \) there is \( \{D_i\}_{i \in I} \subset \Ob(\D) \) with the property that for all \( D \in \Ob(\D) \) and any \( f: C \to G(D) \) there exist \( i \in I \), \( \varphi: G \to G(D_i) \), \( \bar{f}: D_i \to D \) such that \( f = G(\bar{f}) \circ \varphi \).
\end{theorem*}
\begin{proof}
	Clear, since this is equivalent to having a weakly initial set.
\end{proof}

\chapter{Abelian Categories}

\section{Additive Categories}

\begin{definitions*}
	\item An \( \Ab \)-category (a \emdef{preadditive} category) is a category \( \A \) such that
		\begin{enumerate}
			\item \( \A(X, Y) \) is an abelian group for all \( X, Y \in \A \).
			\item The composition map \( \A(X, Y) \times \A(Y, Z) \to \A(X, Z) \) is bilinear:
				\[
					(g_1 + g_2) \circ f = g_1 \circ f + g_2 \circ f.
				\]
		\end{enumerate}
	\item If \( \A \) and \( \B \) are \( \Ab \)-categories, a functor \( F: \A \to \B \) is called \emdef{additive} if
		\[
			\A(X, Y) \to \B(F(X), F(Y))
		\]
		is \( \bb{Z} \)-linear for any \( X, Y \in \A \).
\end{definitions*}

\begin{examples*}
	\item \( \ctRMod \). In particular, \( \ctAb \) and \( \ctVect(k) \).
	\item \( \ctSh(X, R) \).
	\item For \( X \) a ringed space, \( \ctMod(\ca{O}_X) \).
	\item \( \ctQCoh(X) \) and \( \ctCoh(X) \).
\end{examples*}

\begin{proposition*}
	For \( Z \in \A \) the following are equivalent:
	\begin{enumerate}
		\item \( Z \) is initial;
		\item \( Z \) is final;
		\item \( 1_Z = 0_Z \);
		\item \( \A(Z, Z) = \{0_Z\} \).
	\end{enumerate}
\end{proposition*}
\begin{proof}
	Clear.
\end{proof}

\begin{definition*}
	Let \( \A \) be an \( \Ab \)-category and \( X, Y \in \A \). A \emdef{biproduct} of \( X \) and \( Y \) is a diagram
	\[
		\begin{tikzcd}
			X && {X \oplus Y} && Y
			\arrow["{i_1}"', shift right, from=1-1, to=1-3]
			\arrow["{p_1}"', shift right, from=1-3, to=1-1]
			\arrow["{p_2}", shift left, from=1-3, to=1-5]
			\arrow["{i_2}", shift left, from=1-5, to=1-3]
		\end{tikzcd}
	\]
	such that:
	\begin{enumerate}
		\item \( p_1 i_1 = 1_X \);
		\item \( p_2 i_2 = 1_Y \);
		\item \( 1_{X \oplus Y} = i_1 p_1 + i_2 p_2 \).
	\end{enumerate}
	Note that it implies \( p_1 i_2 = 0 = p_2 i_1 \).
\end{definition*}

\begin{theorem*}
	Let \( X, Y \in \A \). The following are equivalent:
	\begin{enumerate}
		\item \( X \) and \( Y \) have a biproduct.
		\item There exists \( X \times Y \) and \( X \times Y \cong X \oplus Y \) with projections \( p_1 \) and \( p_2 \).
		\item There exists \( X \coprod Y \) and \( X \coprod Y \cong X \oplus Y \) with inclusions \( i_1 \) and \( i_2 \).
	\end{enumerate}
\end{theorem*}
\begin{proof}
	We first show \( (1) \To (2) \). Indeed,
	\[
		p_1 i_2 = p_1 (i_1 p_1 + i_2 p_2) i_2 = p_1 i_2 + p_1 i_2 \implies p_1 i_2 = 0.
	\]
	Similarly, \( p_2 i_1 = 0 \).
	\[
		\begin{tikzcd}
			&& T \\
			\\
			X && {X \oplus Y} && Y
			\arrow["{t_1}"', from=1-3, to=3-1]
			\arrow["{\exists! h}"', dashed, from=1-3, to=3-3]
			\arrow["{t_2}", from=1-3, to=3-5]
			\arrow["{p_1}", from=3-3, to=3-1]
			\arrow["{p_2}"', from=3-3, to=3-5]
		\end{tikzcd}
	\]
	Let \( h = i_1 t_1 + i_2 t_2 \). Then \( p_1 h = t_1 \) and \( p_2 h = t_2 \). For any other \( h': T \to X \otimes Y \) such that \( p_i h' = t_i \), we get
	\[
		h' = (i_1 p_1 + i_2 p_2) h' = i_1 t_1 + i_2 t_2 = h.
	\]
	We will only show \( (2) \To (1) \). Inclusions \( i_1 \) and \( i_2 \) can be obtained by the universal property in the following way:
	\[
		\begin{tikzcd}
			&& X \\
			\\
			X && {X \times Y} && Y
			\arrow["{1_X}"', from=1-3, to=3-1]
			\arrow["{\exists! i_1}"', dashed, from=1-3, to=3-3]
			\arrow["0", from=1-3, to=3-5]
			\arrow["{p_1}", from=3-3, to=3-1]
			\arrow["{p_2}"', from=3-3, to=3-5]
		\end{tikzcd}
	\]
	We get that \( p_1 i_1 = 1_X \) and \( p_2 i_2 = 1_Y \). Lastly,
	\[
		\begin{tikzcd}
			&& {X \times Y} \\
			\\
			X && {X \times Y} && Y
			\arrow["{p_1}"', from=1-3, to=3-1]
			\arrow["{\exists!}"', dashed, from=1-3, to=3-3]
			\arrow["{p_2}", from=1-3, to=3-5]
			\arrow["{p_1}", from=3-3, to=3-1]
			\arrow["{p_2}"', from=3-3, to=3-5]
		\end{tikzcd}
	\]
	and \( i_1 p_1 + i_2 p_2 = 1_{X \times Y} \), since
	\[
		p_i (i_1 p_1 + i_2 p_2) = p_i.
	\]
\end{proof}

\begin{definition*}
	An \emdef{additive} category is an \( \Ab \)-category with a zero object and binary biproducts.
\end{definition*}

\begin{proposition*}
	If \( \A \) is additive and \( f, f' \in \A(X, Y) \)
	\[
		f+f' = \nabla_Y (f \sqcup f') \Delta_X,
	\]
	where \( \Delta_X: X \to X \times X \) and \( \nabla: Y \times Y \to Y \).
\end{proposition*}
\begin{proof}
	\textsc{Exercise}.
\end{proof}

\begin{proposition*}
	If \( \A \) and \( \B \) are additive, then \( F: \A \to \B \) is additive if and only if \( F \) preserves biproducts.
\end{proposition*}
\begin{proof}
	\textsc{Exercise}.
\end{proof}

We write
\[
	\ctFunc^{\text{add}}(\A, \B) = (\A, \B).
\]

\section{Abelian Categories}

Let \( \A \) be an additive category with kernels and cokernels.

\[
	\begin{tikzcd}{\ker f} && X &&& Y && {\coker f} \\
		\\
		&& {\coim f} &&& {\im f}
		\arrow["k"', from=1-1, to=1-3]
		\arrow["f", from=1-3, to=1-6]
		\arrow[from=1-3, to=3-3]
		\arrow["{\tilde{f}}", dashed, from=1-3, to=3-6]
		\arrow["c"', from=1-6, to=1-8]
		\arrow["{\bar{f}}"', dotted, from=3-3, to=3-6]
		\arrow[from=3-6, to=1-6]
	\end{tikzcd}
\]
We define \( \im f := \ker c \) and \( \coim f := \coker k \).

\begin{definition*}
	An additive category is called \emdef{abelian} if
	\begin{enumerate}
		\item[AB1] Any map has a kernel and a cokernel.
		\item[AB2] For any map \( f \) the induced map \( \bar{f} \) is an isomorphism.
	\end{enumerate}
\end{definition*}

\begin{remarks*}
	\item \( \A \) is abelian if and only if \( \A^\op \) is abelian.
	\item Any abelian category is finitely (co)complete.
\end{remarks*}

For a complex \( X' \tox{f} X \tox{g} X'' \) with \( gf = 0 \), one has
\[
	\begin{tikzcd}{X'} && X && {X''} \\
		\\
		& {\im f} && {\ker g}
		\arrow["f"', from=1-1, to=1-3]
		\arrow[two heads, from=1-1, to=3-2]
		\arrow["g"', from=1-3, to=1-5]
		\arrow[hook, from=3-2, to=1-3]
		\arrow["\varphi", dashed, from=3-2, to=3-4]
		\arrow[hook', from=3-4, to=1-3]
	\end{tikzcd}
\]
Note that \( \varphi \) is monic. We let \( H = \coker \varphi \).

\begin{definition*}
	A \emdef{cochain complex} in \( \A \) is
	\[
		\begin{tikzcd}{X^\bullet=} & \dotsc && {X^{i - 1}} && {X^i} && {X^{i + 1}} && \dotsc
			\arrow[from=1-2, to=1-4]
			\arrow["{d^{i - 1}}", from=1-4, to=1-6]
			\arrow["{d^i}", from=1-6, to=1-8]
			\arrow[from=1-8, to=1-10]
		\end{tikzcd}
	\]
	with the property that \( d_{i} d_{i - 1} = 0 \) for each \( i \in \bb{Z} \).
\end{definition*}

We define
\[
	H^i(X^\bullet) = H(X^{i - 1} \to X^i \to X^{i + 1}).
\]

\begin{definition*}
	\( X^\bullet \) is called an \emdef{acyclic complex} (or an \emdef{exact sequence}) if \( H^i(X^\bullet) = 0 \).
\end{definition*}

\begin{examples*}
	\item \( 0 \to A \tox{f} B \) is exact if and only if \( f \) is monic.
	\item \( A \tox{f} B \to 0 \) is exact  if and only if \( f \) is epic.
	\item \( 0 \to A \tox{f} B \tox{g} C \to 0 \) (which is called a \emdef{short exact sequence}, a SES) is exact if and only if \( f \) is monic and \( C = \coker f \) if and only if \( g \) is epic and \( A = \ker f \).
\end{examples*}

\begin{definition*}
	A short exact sequence \emdef{splits} if there is \( \varphi: X \to X' \oplus X'' \) such that
	\[
		\begin{tikzcd}
			0 && {X'} && X && {X''} && 0 \\
			\\
			0 && {X'} && {X' \oplus X''} && {X''} && 0
			\arrow[from=1-1, to=1-3]
			\arrow[from=1-3, to=1-5]
			\arrow["{=}"', from=1-3, to=3-3]
			\arrow[from=1-5, to=1-7]
			\arrow["\varphi", from=1-5, to=3-5]
			\arrow[from=1-7, to=1-9]
			\arrow["{=}", from=1-7, to=3-7]
			\arrow[from=3-1, to=3-3]
			\arrow[from=3-3, to=3-5]
			\arrow[from=3-5, to=3-7]
			\arrow[from=3-7, to=3-9]
		\end{tikzcd}
	\]
	is commutative.
\end{definition*}

\begin{proposition*}
	For a short exact sequence, the following are equivalent:
	\begin{enumerate}
		\item it splits;
		\item there is \( \sigma: X'' \to X \) such that \( g \sigma = 1_{X''} \);
		\item there is \( \rho: X \to X' \) such that \( \rho f = 1_{X'} \).
	\end{enumerate}
\end{proposition*}
\begin{proof}
	\textsc{Exercise}.
\end{proof}

\section{Projective and Injective Objects}

Let \( \A \) and \( \B \) be abelian categories and \( F: \A \to \B \) be a functor.

\begin{definitions*}
	\item \( F \) is \emdef{left exact} (Lex) if for any exact sequence \( 0 \to X' \to X \to X'' \) the sequence
		\[
			0 \to F(X') \to F(X) \to F(X'')
		\]
		is exact.
	\item \( F \) is \emdef{right exact} (Rex) if for any exact sequence \( 0 \to X' \to X \to X'' \) the sequence
		\[
			F(X') \to F(X) \to F(X'') \to 0
		\]
		is exact.
	\item \( F \) is \emdef{exact} if \( F \) is left exact and right exact.
\end{definitions*}

\begin{examples*}
	\item \( B \otimes - \) is right exact.
	\item \( \Hom(B, -) \) is left exact.
\end{examples*}

\begin{definition*}
	An object \( X \in \A \) is \emdef{projective} (corresp. \emdef{injective}) if the functor \( h_X = \Hom(X, -) \) (corresp. \( h^X = \Hom(-, X) \)) is exact.
\end{definition*}

\newpage

\begin{theorem*}
	Let \( P \in \ctRMod \). The following are equivalent:
	\begin{enumerate}
		\item \( P \) is projective.
		\item For any epimorphism \( \beta: M \to N \) and any \( f: P \to N \) there is \( \sigma: P \to M \) such that \( \beta \sigma = f \).
			\[
				\begin{tikzcd}
					M && N && 0 \\
					\\
					&& P
					\arrow["\beta", from=1-1, to=1-3]
					\arrow[from=1-3, to=1-5]
					\arrow["\sigma", dashed, from=3-3, to=1-1]
					\arrow["f"', from=3-3, to=1-3]
				\end{tikzcd}
			\]
		\item Any short exact sequence \( 0 \to K \to M \to P \to 0 \) splits.
		\item There is \( K \in \ctRMod \) such that \( P \oplus K \cong R^{(S)} \).
	\end{enumerate}
\end{theorem*}
\begin{proof}
	\textsc{For next time}.
\end{proof}